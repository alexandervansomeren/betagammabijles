\documentclass{article}
\usepackage{amsfonts}
\usepackage{enumerate}
\usepackage{comment}
\usepackage[english]{babel}

\begin {document}

\title{Verslag WEBDB13BG2}
\author{Bas Nachtegaal\\
    Laura Helgering\\
    Tim Leunissen\\
    Alexander van Someren\\
    Emma Boumans}
\date{januari 2013}
\maketitle

\begin{comment}
Dit is een comment waarin we de voortgang van het verslag kunnen bijhouden. Als je iets update, is het handig als je dat hier even bij zet. Ik (Emma) ben zojuist begonnen aan het verslag.
\end{comment}

\section{Inleiding}
Dit wordt de LateX file van ons verslag\\
Zet in de hoofd-directory van de zipfile een PDF versie van het verslag. Het verslag bestaat uit twee delen. In het eerste deel worden ontwerpkeuzen toegelicht voor de klant (vertel waarom je produkt zo goed is). Het tweede deel is geschreven voor de ICT-afdeling van de klant. Dit vertelt hoe de site technisch is gerealiseerd, en dient als installatie- en onderhoudshandleiding. \\
Documentatie op details (afzonderlijke functies) mag in de code. Dat hoeft dus niet in het verslag! Spreek af hoe het schrijfwerk verdeeld wordt en wie de credits ervoor krijgt (krijgen). 

\end{document}
